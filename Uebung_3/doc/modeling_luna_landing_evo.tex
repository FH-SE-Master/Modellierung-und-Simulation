\section{Optimierung mit einen Evolutionsalgorithmus}
\subsection{Der Evolutionsalgorithmus}
Folgende Quelltexte sind das implementierte Hauptprogramm und die implementierten Funktionen des Evolutionsalgorithmus.

\begin{code}
	\caption{Hauptprogramm}
	\mSourceFile{\srcDir/moonLanding.m}
	\label{fig:moon-landing-m}
\end{code}

\begin{code}
	\caption{Funktion die einen Lösungskandidaten initialisiert}
	\mSourceFile{\srcDir/initialize.m}
	\label{fig:initialize-m}
\end{code}

\begin{code}
\caption{Funktion, welche die Mutanten aus einem Elter erzeugt}
\mSourceFile{\srcDir/bread.m}
\label{fig:bread-m}
\end{code}

\begin{code}
\caption{Funktion, welche einen Elter zu einem Mutanten konvertiert}
\mSourceFile{\srcDir/mutate.m}
\label{fig:mutate-m}
\end{code}

\begin{code}
\caption{Funktion, welche die Simulation für einen Lösungskandidaten durchführt}
\mSourceFile{\srcDir/evaluate.m}
\label{fig:evaluate-m}
\end{code}
\newpage

\subsection{Auswertung der Testfälle}
\subsubsection{Test 1}

\subsubsection{Test 2}

\subsubsection{Test 3}

\subsubsection{Test 4}

\subsubsection{Test 5}

\subsubsection{Gegenüberstellung der Tests}
\newpage

\subsection{Optimierungen des Evolutionsalgorithmus}
\subsubsection{Was muss optimiert werden ?}

\subsubsection{Was ist eine geeignete Fitnessfunktion ?}

\subsubsection{Welche Parameter können modifiziert werden ?}

\subsubsection{Warum ein Evolutionsalgorithmus ?}
