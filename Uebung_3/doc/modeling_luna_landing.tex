% Section gramar and basics 
\section{Modellierung einer Mondlandung}
\label{sec:modeling-luna-landing}

\subsection{Simulink Modell der Mondlandung}
\label{sec:sub-simulink-model-luna-landing}
Die Abbildung \ref{fig:simulink-model-luna-landing} zeigt das implementierte Modell der Mondlandung.
\begin{figure}[h]
	\centering
	\includegraphics[scale=0.6]{\imageDir/simulink-model-luna-landing.JPG}
	\caption{Simulink Modell der Mondlandung}
	\label{fig:simulink-model-luna-landing}
\end{figure}

\subsubsection{Test 1}
\begin{figure}[h]
	\centering
	\includegraphics[scale=0.3]{\imageDir/simulink-luna-landing-test-1.JPG}
	\caption{$h_{(t)}$=~-0.0088 $m$, $v_{(t)}$=~-7.17 $m/s$}
	\label{fig:simulink-luna-landing-test-1}
\end{figure}
\ \newpage

\subsubsection{Test 2}
\begin{figure}[h]
	\centering
	\includegraphics[scale=0.3]{\imageDir/simulink-luna-landing-test-2.JPG}
	\caption{$h_{(t)}$=~-0.0256 $m$, $v_{(t)}$=~-8.72 $m/s$}
	\label{fig:simulink-luna-landing-test-2}
\end{figure}

\subsubsection{Test 3}
\begin{figure}[h]
	\centering
	\includegraphics[scale=0.3]{\imageDir/simulink-luna-landing-test-3.JPG}
	\caption{$h_{(t)}$=~-0.0018 $m$, $v_{(t)}$=~-1.44 $m/s$}
	\label{fig:simulink-luna-landing-test-3}
\end{figure}
\ \newpage

\subsubsection{Testergebnisse}
\label{sec:simulink-luna-landing-test-results}
\begin{figure}[h]
	\centering
	\includegraphics[scale=0.8]{\imageDir/simulink-luna-landing-test-results.JPG}
	\caption{Tabelle mit den Gesamtergebnissen}
	\label{fig:simulink-luna-landing-test-result}
\end{figure}


\subsection{Solver Modifikationen}
\label{sec:sub-simulink-solver-luna-landing}
Als Beispiel wird der Lösungskandidat aus Abschnitt \ref{sec:simulink-luna-landing-test-results} herangezogen, mit dem gezeigt wird, wie sich die Änderungen an den Einstellungen auswirken.

\subsubsection{Test Schrittweite 1 und 0.01}
\begin{figure}[h]
	\centering
	\includegraphics[scale=0.8]{\imageDir/simulink-luna-landing-solver-test-1.JPG}
	\caption{Testergebnisse mit den drei Schrittweiten 0.01, 0.1 und 1}
	\label{fig:simulink-luna-landing-solver-test-1-1}
\end{figure}
\ \newline
Die gravierenden Unterschiede entstehen, da mit einer zu großen Schrittweite schnelle Veränderungen im System nicht erkannt werden können. Bei schnellen Veränderungen im System ist eine kleine Schrittweite von Vorteil, bei langsamen Veränderungen eine große Schrittweite.

\subsubsection{Test verschiedener Integrationsmethoden}
\begin{figure}[h]
	\centering
	\includegraphics[scale=0.8]{\imageDir/simulink-luna-landing-solver-test-2.JPG}
	\caption{Testergebnisse mit den vier Integrationsmethoden}
	\label{fig:simulink-luna-landing-solver-test-1-1}
\end{figure}
\ \newline
Die verschiedenen Integrationsmethoden \emph{ode5}, \emph{euler} unterscheiden sich durch ihren lokalen und globalen Error. Diese Error Indikatoren werden  durch verschiedene Mechanismen die bei der Integration angewandt werden, wie gewichtetes Mittel, Miteinbeziehung von vorherigen und/oder geschätzten Nachfolgern und der Verwendung empirischer Faktoren, beeinflusst.
\newpage

\begin{code}
	\caption{Testprogramm für die Simulation mit modifiziertem Solver}
	\mSourceFile{\srcDir/lunaLandingSolver.m}
	\label{fig:luna-landing-solver-m}
\end{code}

\newpage

\subsection{Modifikation der Mondlandungsmodell}
\label{sec:sub-simulink-modified-model-luna-landing}
Die Abbildung \ref{fig:simulink-luna-landing-modified} zeigt das modifizierte Modell der Mondlandung.
\begin{figure}[h]
	\centering
	\includegraphics[scale=0.7]{\imageDir/simulink-model-luna-landing-modified.JPG}
	\caption{Mondladungsmodell mit Massenreduktion beim Bremsen}
	\label{fig:simulink-luna-landing-modified}
\end{figure}

\subsubsection{Test 1}
\begin{figure}[h]
	\centering
	\includegraphics[scale=0.3]{\imageDir/simulink-luna-landing-modified-test-1.JPG}
	\caption{$h_{(t)}$=~-0.0685 $m$, $v_{(t)}$=~-6.91 $m/s$}
	\label{fig:simulink-luna-landing-modified-test-1}
\end{figure}
\ \newpage

\subsubsection{Test 2}
\begin{figure}[h]
	\centering
	\includegraphics[scale=0.3]{\imageDir/simulink-luna-landing-modified-test-2.JPG}
	\caption{$h_{(t)}$=~-0.033 $m$, $v_{(t)}$=~-8.088 $m/s$}
	\label{fig:simulink-luna-landing-modified-test-2}
\end{figure}

\subsubsection{Test 3}
\begin{figure}[h]
	\centering
	\includegraphics[scale=0.3]{\imageDir/simulink-luna-landing-modified-test-3.JPG}
	\caption{$h_{(t)}$=~-0.0088 $m$, $v_{(t)}$=~-3.4123 $m/s$}
	\label{fig:simulink-luna-landing-modified-test-3}
\end{figure}
\newpage

\subsubsection{Testergebnisse}
\label{sec:simulink-luna-landing-modified-test-results}
\begin{figure}[h]
	\centering
	\includegraphics[scale=0.65]{\imageDir/simulink-luna-landing-modified-test-results.JPG}
	\caption{Tabelle mit den Gesamtergebnissen}
	\label{fig:simulink-luna-landing-modified-test-result}
\end{figure}
\ \newline
Es wurden dieselben Parametervektoren wie für das originale Modell verwendet, wobei sich die Resultate für die ersten beiden Testfälle verbessert und für den dritten Testfall akzeptabel verschlechtert haben.
\newpage