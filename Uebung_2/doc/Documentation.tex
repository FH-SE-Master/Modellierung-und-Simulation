\documentclass[11pt, a4paper, twoside]{article}   	% use "amsart" instead of "article" for AMSLaTeX format

\usepackage{geometry}                		% See geometry.pdf to learn the layout options. There are lots.
\usepackage{pdfpages}
\usepackage{caption}
\usepackage{minted}
\usepackage[german]{babel}			% this end the next are needed for german umlaute
\usepackage[utf8]{inputenc}
\usepackage{color}
\usepackage{graphicx}
\usepackage{titlesec}
\usepackage{fancyhdr}
\usepackage{lastpage}
\usepackage{hyperref}
\usepackage[autostyle=false, style=english]{csquotes}
\usepackage{mathtools}
\usepackage{tabularx}
% http://www.artofproblemsolving.com/wiki/index.php/LaTeX:Symbols#Operators
% =============================================
% Layout & Colors
% =============================================
\geometry{
   a4paper,
   total={210mm,297mm},
   left=20mm,
   right=20mm,
   top=20mm,
   bottom=30mm
 }	

\definecolor{myred}{rgb}{0.8,0,0}
\definecolor{mygreen}{rgb}{0,0.6,0}
\definecolor{mygray}{rgb}{0.5,0.5,0.5}
\definecolor{mymauve}{rgb}{0.58,0,0.82}

\setcounter{secnumdepth}{4}


% the default java directory structure and the main packages
\newcommand{\srcDir}{../src/}
\newcommand{\imageDir}{./images/}
% =============================================
% Code Settings
% =============================================
\newenvironment{code}{\captionsetup{type=listing}}{}
\newmintedfile[mSourceFile]{matlab}{
	linenos=true, 
	frame=single, 
	breaklines=true, 
	tabsize=2,
	numbersep=5pt,
	xleftmargin=10pt,
	baselinestretch=1,
	fontsize=\footnotesize
}
\newmintinline[mInlineSource]{matlab}{}
\newminted[mSource]{matlab}{
	breaklines=true, 
	tabsize=2,
	autogobble=true,
	breakautoindent=false
}
% =============================================
% Page Style, Footers & Headers, Title
% =============================================
\title{Übung 1}
\author{Thomas Herzog}

\lhead{Übung 1}
\chead{}
\rhead{\includegraphics[scale=0.10]{FHO_Logo_Students.jpg}}

\lfoot{S1610454013}
\cfoot{}
\rfoot{ \thepage / \pageref{LastPage} }
\renewcommand{\footrulewidth}{0.4pt}
% =============================================
% D O C U M E N T     C O N T E N T
% =============================================
% =============================================
% 2016.10.13: 1 
% 2016.10.14: 2
% =============================================

\pagestyle{fancy}
\begin{document}
\setlength{\headheight}{15mm}
\includepdf[pages={1,2}]{Uebungszettel02.pdf}

% Section gramar and basics 
\section{Kontinuierliche Modellierung}
\label{sec:continous-modeling}
Dieser Abschnitt beschäftigt sich mit der Aufgabenstellung 1 der zweiten Übung. 

\subsection{Erstes Blockdiagramm}
Dieser Abschnitt beschäftigt ich mit dem Aufstellen der Gleichungen in A, B, C Normalform, die aus den gegebenen Blockdiagramm abgeleitet wurden.
\newline
\newline
$U= \begin{bmatrix}
	u_1 \\[0.3em]
	u_2 \\[0.3em]
\end{bmatrix}
,
X = \begin{bmatrix}
	z \\[0.3em]
	v \\[0.3em]
	x \\[0.3em]
\end{bmatrix}
,
Y=\begin{bmatrix}
	y_1 \\[0.3em]
	y_2 \\[0.3em]
	y_3 \\[0.3em]
\end{bmatrix}
\newline
\newline
\newline
z'= 0*z + 0*v + 0*x + 2*u_1 + 2*u_2 \hspace{2mm} \equiv \hspace{2mm} 2*u_1 + 2*u_2
\newline
v'= 1*z + 0*v + 0*x + 0*u_1 - 1*u_2 \hspace{2mm} \equiv \hspace{2mm} z - u_2
\newline
x'= 0*z + 1*v + 0*x + 0*u_1 + 0*u_2 \hspace{2mm} \equiv \hspace{2mm} v 
\newline
\newline
y_1 = 0*z + 0*v + 0* x \hspace{30mm} \equiv \hspace{2mm} 0
\newline
y_2 = 0*z + 1*v + 0* x \hspace{30mm} \equiv \hspace{2mm} v
\newline
y_3 = 0*z + 0*v + 1* x \hspace{30mm} \equiv \hspace{2mm} x
\newline
\newline
\newline
X'=A*X + B * U
\newline
\newline
X'= \begin{bmatrix}
	0 & 0 & 0 \\[0.3em]
	1 & 0 & 0 \\[0.3em]
	0 & 1 & 0 \\[0.3em]
\end{bmatrix}
* 
\begin{bmatrix}
	z \\[0.3em]
	v \\[0.3em]
	x \\[0.3em]
\end{bmatrix}
+
\begin{bmatrix}
	2 & 2\\[0.3em]
	0 & -1 \\[0.3em]
	0 & 0 \\[0.3em]
\end{bmatrix}
*
\begin{bmatrix}
	u_1 \\[0.3em]
	u_2 \\[0.3em]
\end{bmatrix}
\newline
\newline
\newline
\newline
Y=C*X
\newline
\newline
Y=
\begin{bmatrix}
	0 & 0 & 0 \\[0.3em]
	0 & 1 & 0 \\[0.3em]
	0 & 0 & 1 \\[0.3em]
\end{bmatrix}
*
\begin{bmatrix}
	z \\[0.3em]
	v \\[0.3em]
	x \\[0.3em]
\end{bmatrix}
$
\newpage

\subsection{Zweites Blockdiagramm}
Dieser Abschnitt beschäftigt ich mit dem Aufstellen der Gleichungen in A, B, C Normalform, die aus den gegebenen Blockdiagramm abgeleitet wurden.
\newline
\newline
$U= \begin{bmatrix}
	u_1 \\[0.3em]
	u_2 \\[0.3em]
	u_3 \\[0.3em]
\end{bmatrix}
,
X = \begin{bmatrix}
	z \\[0.3em]
	x \\[0.3em]
\end{bmatrix}
,
Y = \begin{bmatrix}
	y1 \\[0.3em]
	y2 \\[0.3em]
\end{bmatrix}
\newline
\newline
\newline
z'= 0*z + 0*x + 0*u_1 - 5*u_2 + 0*u_3\hspace{15mm} \equiv \hspace{2mm} - 5*u_2
\newline
x'= 1*z + 1*x + 0*u_1 + (-2*u_2 + 1) + 1*u_3 \hspace{2mm} \equiv \hspace{2mm} z + x + (-2*u_2 + 1) + u_3
\newline
\newline
y_1 = 0*z + 0* x \hspace{57mm} \equiv \hspace{2mm} 0
\newline
y_2 = 0*z + 1* x \hspace{57mm} \equiv \hspace{2mm} x
\newline
\newline
\newline
X'=A*X + B * U
\newline
\newline
X'= \begin{bmatrix}
	0 & 0 \\[0.3em]
	1 & 1 \\[0.3em]
\end{bmatrix}
* 
\begin{bmatrix}
	z \\[0.3em]
	x \\[0.3em]
\end{bmatrix}
+
\begin{bmatrix}
	0 & -5 & 0 \\[0.3em]
	0 & (-2*u_2 + 1) & 1 \\[0.3em]
\end{bmatrix}
*
\begin{bmatrix}
	u_1 \\[0.3em]
	u_2 \\[0.3em]
	u_3 \\[0.3em]
\end{bmatrix}
\newline
\newline
\newline
\newline
Y=C*X
\newline
\newline
Y =
C = \begin{bmatrix}
	0 & 0 \\[0.3em]
	0 & 1 \\[0.3em]
\end{bmatrix}
*
\begin{bmatrix}
	z \\[0.3em]
	x \\[0.3em]
\end{bmatrix}
$
\newpage

\subsection{Erstes System als Blockschaltbild}
$
U= \begin{bmatrix}
	u_1 \\[0.3em]
	u_2 \\[0.3em]
\end{bmatrix}
,
X = \begin{bmatrix}
	x \\[0.3em]
	v \\[0.3em]
	z \\[0.3em]
\end{bmatrix}
$
\begin{figure}[h]
\centering
\includegraphics[scale=0.70,angle=90]{\imageDir/aufgabe_b1.JPG}
\caption{Blockschaltbild zum System aus Aufgabe b1}
\label{fig:exercise-b1}
\end{figure}
\newpage

\subsection{Zweites System als Blockschaltbild}
$
U= \begin{bmatrix}
	u_1 \\[0.3em]
	u_2 \\[0.3em]
\end{bmatrix}
,
X = \begin{bmatrix}
	x \\[0.3em]
	v \\[0.3em]
\end{bmatrix}
\newline
\newline
\newline
$
\begin{figure}[h]
\centering
\includegraphics[scale=0.8,angle=90]{\imageDir/aufgabe_b2.JPG}
\caption{Blockschaltbild zum System aus Aufgabe b2}
\label{fig:exercise-b1}
\end{figure}
\newpage

\subsection{Fehlerhaftes System in A, B, C Normalform}
An diesem System ist die $C$-Matrix falsch, da dieses System drei Systemzustände besitzt und daher die $C$-Matrix drei Spalten benötigt, so viele wie es Systemzustände gibt.
\newline
\newline
$
Y=C*X \hspace{5mm} wobei \hspace{3mm} 
X= \begin{bmatrix}
	x \\[0.3em]
	v \\[0.3em]
	z \\[0.3em]
\end{bmatrix}
\hspace{3mm} und \hspace{3mm} 
C = \begin{bmatrix}
	1 & 0 & 0 \\[0.3em]
	2 & 0 & 0 \\[0.3em]
	3 & 0 & 0 \\[0.3em]
\end{bmatrix}
$

\subsection{System als Blockschaltbild und in A, B, C Normalform}
$
X = \begin{bmatrix}
	a \\[0.3em]
	b \\[0.3em]
	c \\[0.3em]
\end{bmatrix}
,
U = \begin{bmatrix}
	i_1 \\[0.3em]
	i_2 \\[0.3em]
\end{bmatrix}
\newline
\newline
\newline
X'=A*X+B*U
\newline
\newline
X'= \begin{bmatrix}
	2 & 4 & -2 \\[0.3em]
	0 & 4 & -1 \\[0.3em]
	0 & 0 & -1 \\[0.3em]
\end{bmatrix}
*
\begin{bmatrix}
	a \\[0.3em]
	b \\[0.3em]
	c \\[0.3em]
\end{bmatrix}
+
\begin{bmatrix}
	2 & 0 \\[0.3em]
	0 & 1 \\[0.3em]
	3 & 0 \\[0.3em]
\end{bmatrix}
*
\begin{bmatrix}
	i_1 \\[0.3em]
	i_2 \\[0.3em]
\end{bmatrix}
\newline
\newline
\newline
\newline
Y=C*X
\newline
\newline
Y=
\begin{bmatrix}
	3 & 0 & 0 \\[0.3em]
	0 & -1 & 0 \\[0.3em]
	1 & 0 & -1 \\[0.3em]
\end{bmatrix}
*
\begin{bmatrix}
	a \\[0.3em]
	b \\[0.3em]
	c \\[0.3em]
\end{bmatrix}
=
\begin{bmatrix}
	(3*a) \\[0.3em]
	(-b) \\[0.3em]
	(a-c) \\[0.3em]
\end{bmatrix}
$

\begin{figure}
\centering
\includegraphics[scale=0.7,angle=90]{\imageDir/aufgabe_b3.JPG}
\caption{Blockschaltbild zum System aus Aufgabe b3}
\label{fig:exercise-b1}
\end{figure}
\ \newpage

\section{Kontinuierliche Simulation}
\label{sec:continous-simulation}


\end{document}